% Options for packages loaded elsewhere
\PassOptionsToPackage{unicode}{hyperref}
\PassOptionsToPackage{hyphens}{url}
%
\documentclass[
]{article}
\usepackage{amsmath,amssymb}
\usepackage{lmodern}
\usepackage{iftex}
\ifPDFTeX
  \usepackage[T1]{fontenc}
  \usepackage[utf8]{inputenc}
  \usepackage{textcomp} % provide euro and other symbols
\else % if luatex or xetex
  \usepackage{unicode-math}
  \defaultfontfeatures{Scale=MatchLowercase}
  \defaultfontfeatures[\rmfamily]{Ligatures=TeX,Scale=1}
\fi
% Use upquote if available, for straight quotes in verbatim environments
\IfFileExists{upquote.sty}{\usepackage{upquote}}{}
\IfFileExists{microtype.sty}{% use microtype if available
  \usepackage[]{microtype}
  \UseMicrotypeSet[protrusion]{basicmath} % disable protrusion for tt fonts
}{}
\makeatletter
\@ifundefined{KOMAClassName}{% if non-KOMA class
  \IfFileExists{parskip.sty}{%
    \usepackage{parskip}
  }{% else
    \setlength{\parindent}{0pt}
    \setlength{\parskip}{6pt plus 2pt minus 1pt}}
}{% if KOMA class
  \KOMAoptions{parskip=half}}
\makeatother
\usepackage{xcolor}
\usepackage[margin=1in]{geometry}
\usepackage{graphicx}
\makeatletter
\def\maxwidth{\ifdim\Gin@nat@width>\linewidth\linewidth\else\Gin@nat@width\fi}
\def\maxheight{\ifdim\Gin@nat@height>\textheight\textheight\else\Gin@nat@height\fi}
\makeatother
% Scale images if necessary, so that they will not overflow the page
% margins by default, and it is still possible to overwrite the defaults
% using explicit options in \includegraphics[width, height, ...]{}
\setkeys{Gin}{width=\maxwidth,height=\maxheight,keepaspectratio}
% Set default figure placement to htbp
\makeatletter
\def\fps@figure{htbp}
\makeatother
\setlength{\emergencystretch}{3em} % prevent overfull lines
\providecommand{\tightlist}{%
  \setlength{\itemsep}{0pt}\setlength{\parskip}{0pt}}
\setcounter{secnumdepth}{-\maxdimen} % remove section numbering
\ifLuaTeX
  \usepackage{selnolig}  % disable illegal ligatures
\fi
\IfFileExists{bookmark.sty}{\usepackage{bookmark}}{\usepackage{hyperref}}
\IfFileExists{xurl.sty}{\usepackage{xurl}}{} % add URL line breaks if available
\urlstyle{same} % disable monospaced font for URLs
\hypersetup{
  hidelinks,
  pdfcreator={LaTeX via pandoc}}

\author{}
\date{\vspace{-2.5em}}

\begin{document}

\hypertarget{fall_flow_redd_dewatering}{%
\section{Fall\_Flow\_Redd\_Dewatering}\label{fall_flow_redd_dewatering}}

code to create one page summary of Fall Flow scenarios and shallow redd
dewatering \# LTO DSM Wrapper A wrapper for running WINTER RUN CHINOOK
SALMON SCIENCE INTEGRATION TEAM MODEL and CHINOOK SALMON SCIENCE
INTEGRATION TEAM MODEL as part of the Long-term Operation of the CVP.
\textbf{September 2023}

\hypertarget{primary-authors}{%
\subsection{Primary Authors:}\label{primary-authors}}

Lisa Elliott U.S. Bureau of Reclamation, Bay-Delta Office, Science
Division \href{mailto:lelliott@usbr.gov}{\nolinkurl{lelliott@usbr.gov}}

Chase Ehlo U.S. Bureau of Reclamation, Bay-Delta Office, Science
Division \href{mailto:cehlo@usbr.gov}{\nolinkurl{cehlo@usbr.gov}}

\hypertarget{disclaimers}{%
\subsection{Disclaimers:}\label{disclaimers}}

The script used to generate this document was developed by Bureau of
Reclamation for the purposes of providing decision support for real-time
Fall Flow reductions. The resulting document is for the purpose of
comparing proposed alternatives and does not imply a specific policy
position. It is for the purpose of comparing alternatives. No warranty
expressed or implied is made regarding the display or utility of the
code for other purposes, nor on all computer systems, nor shall the act
of distribution constitute any such warranty. The U.S. Bureau of
Reclamation or the U.S. Government shall not be held liable for improper
or incorrect use of the code described and/or contained herein.

\hypertarget{how-to-run-the-code}{%
\subsection{How to run the code:}\label{how-to-run-the-code}}

\textbf{Update} the Redds.csv file with the latest shallow redds data
from Calfish.org or CDFW. Make sure that dates are clearly in
``\%Y-\%m-\%d'' format. \textbf{Update} the kesFlow.csv file with any
new flow scenarios of interest. Make sure that cell formatting for all
flows is ``General'' (no commas). \textbf{Check} for an update on the
YYYY\_WR\_INTERNET\_CARCASS-REDDS\_counts\_as\_of\_M\_dd\_YY.xlsx file
from calfish.org. Check the ``Reporting'' sheet for the ``To date,
unexpanded redd count'' (cell B88 as of 2023).\\
\textbf{Update} input lines for ./Real-time
Estimates\_Prelim\_automated\_v3.Rmd. All inputs that need to be
adjusted are in the first chunk of code. \#\#\# Inputs to update: +
reddCount = To date, unexpanded redd count from
YYYY\_WR\_INTERNET\_CARCASS-REDDS\_counts\_as\_of\_M\_dd\_YY.xlsx *
countDate = date associated with ``To date, unexpanded redd count from
YYYY\_WR\_INTERNET\_CARCASS-REDDS\_counts\_as\_of\_M\_dd\_YY.xlsx'''' *
updatedReddInfoDate = date of most recent shallow redds data from
CDFW/PSMFC/calfish.org * yr = year of analysis * yr\_exp \#the analysis
year's expected expansion number based on the linear relationship
between yearly expansions vs recapture rate of tagged female salmon (see
WR EXPANSIONS for Dewatered Redd Calculations 9-11-23.xlsx from CDFW) *
fallreddCount = a number of fall run redds to use in simulations. We
recommend using the current value of 1000. * scenarios = A list of
scenarios under consideration * kwk\_webpage = cdec website for KWK gage
daily flows * kes\_webpage = cdec website for KES gage daily flows
\textbf{Run} ./Real-time Estimates\_Prelim\_automated\_v3.Rmd with
Run/Run ALL to check that all code runs properly. \textbf{Knit}
./Real-time Estimates\_Prelim\_automated\_v3.Rmd with Knit/Knit to Word

\end{document}
