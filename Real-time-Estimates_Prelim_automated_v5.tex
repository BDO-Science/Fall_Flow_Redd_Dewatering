% Options for packages loaded elsewhere
\PassOptionsToPackage{unicode}{hyperref}
\PassOptionsToPackage{hyphens}{url}
%
\documentclass[
]{article}
\usepackage{amsmath,amssymb}
\usepackage{lmodern}
\usepackage{iftex}
\ifPDFTeX
  \usepackage[T1]{fontenc}
  \usepackage[utf8]{inputenc}
  \usepackage{textcomp} % provide euro and other symbols
\else % if luatex or xetex
  \usepackage{unicode-math}
  \defaultfontfeatures{Scale=MatchLowercase}
  \defaultfontfeatures[\rmfamily]{Ligatures=TeX,Scale=1}
\fi
% Use upquote if available, for straight quotes in verbatim environments
\IfFileExists{upquote.sty}{\usepackage{upquote}}{}
\IfFileExists{microtype.sty}{% use microtype if available
  \usepackage[]{microtype}
  \UseMicrotypeSet[protrusion]{basicmath} % disable protrusion for tt fonts
}{}
\makeatletter
\@ifundefined{KOMAClassName}{% if non-KOMA class
  \IfFileExists{parskip.sty}{%
    \usepackage{parskip}
  }{% else
    \setlength{\parindent}{0pt}
    \setlength{\parskip}{6pt plus 2pt minus 1pt}}
}{% if KOMA class
  \KOMAoptions{parskip=half}}
\makeatother
\usepackage{xcolor}
\usepackage[margin=1in]{geometry}
\usepackage{graphicx}
\makeatletter
\def\maxwidth{\ifdim\Gin@nat@width>\linewidth\linewidth\else\Gin@nat@width\fi}
\def\maxheight{\ifdim\Gin@nat@height>\textheight\textheight\else\Gin@nat@height\fi}
\makeatother
% Scale images if necessary, so that they will not overflow the page
% margins by default, and it is still possible to overwrite the defaults
% using explicit options in \includegraphics[width, height, ...]{}
\setkeys{Gin}{width=\maxwidth,height=\maxheight,keepaspectratio}
% Set default figure placement to htbp
\makeatletter
\def\fps@figure{htbp}
\makeatother
\setlength{\emergencystretch}{3em} % prevent overfull lines
\providecommand{\tightlist}{%
  \setlength{\itemsep}{0pt}\setlength{\parskip}{0pt}}
\setcounter{secnumdepth}{-\maxdimen} % remove section numbering
\usepackage{booktabs}
\usepackage{longtable}
\usepackage{array}
\usepackage{multirow}
\usepackage{wrapfig}
\usepackage{float}
\usepackage{colortbl}
\usepackage{pdflscape}
\usepackage{tabu}
\usepackage{threeparttable}
\usepackage{threeparttablex}
\usepackage[normalem]{ulem}
\usepackage{makecell}
\usepackage{xcolor}
\ifLuaTeX
  \usepackage{selnolig}  % disable illegal ligatures
\fi
\IfFileExists{bookmark.sty}{\usepackage{bookmark}}{\usepackage{hyperref}}
\IfFileExists{xurl.sty}{\usepackage{xurl}}{} % add URL line breaks if available
\urlstyle{same} % disable monospaced font for URLs
\hypersetup{
  pdftitle={PRELIMINARY DATA: Redd Dewatering Estimates for Keswick Fall Flow Scenarios},
  pdfauthor={BDO Science Division},
  hidelinks,
  pdfcreator={LaTeX via pandoc}}

\title{PRELIMINARY DATA: Redd Dewatering Estimates for Keswick Fall Flow
Scenarios}
\author{BDO Science Division}
\date{08 October, 2024}

\begin{document}
\maketitle

This script constructs real-time winter-run redd dewatering estimates
based on most recent data available from CDFW (2024-10-01) for
winter-run data and dewatering estimates from USFWS (2006; see
citation). Data are also available in 2024 Winter-run Data file.xls
online at
\href{https://gcc02.safelinks.protection.outlook.com/?url=https\%3A\%2F\%2Fwww.calfish.org\%2FProgramsData\%2FConservationandManagement\%2FCentralValleyMonitoring\%2FCDFWUpperSacRiverBasinSalmonidMonitoring.aspx\&data=05\%7C01\%7Clelliott\%40usbr.gov\%7C689ebb9a6c8243b4f96c08da90f5c542\%7C0693b5ba4b184d7b9341f32f400a5494\%7C0\%7C0\%7C637981682646098788\%7CUnknown\%7CTWFpbGZsb3d8eyJWIjoiMC4wLjAwMDAiLCJQIjoiV2luMzIiLCJBTiI6Ik1haWwiLCJXVCI6Mn0\%3D\%7C3000\%7C\%7C\%7C\&sdata=A1eQkWPxbkXxnzEvc2K8\%2FTmslZ8H8zvxdks3\%2F78Yrvw\%3D\&reserved=0}{calfish.org}.

This document is also now available on
\href{https://www.cbr.washington.edu/sacramento/workgroups/usst.html\#redd_dewater}{SacPAS}.
However, data on the SacPAS webpage may not immediately reflect data in
the document as updates to the webpage with newest flow and redd data
may be pending.

Please note that all data are preliminary until data collection is
finalized. Likewise, there are uncertainties with forecasts which may
lead to changes in proposed operations.

\hypertarget{current-winter-run-chinook-salmon-redd-count}{%
\section{Current Winter-run Chinook Salmon Redd
Count}\label{current-winter-run-chinook-salmon-redd-count}}

As of September 10, 2024, the unexpanded redd count is \textbf{152}
Winter-run redds. It is important to note that until data collection is
completed for the year these are the \textbf{minimum} number of possible
redds. The Winter-run number will always expand upon final analysis but
gives an in-season guard rail of the minimum number of redds this year.

Given that the number of Winter-run redds is always larger than the
early season carcass counts, an expansion number based on historic data
is multiplied by the carcass count to estimate the total number of redds
for the season before the end of the season's final estimate is
developed and the final redd count is known. Average 2005-2022 expansion
was 1.98 * the total redd count, and thus we focus on an expansion
factor of 2 to represent expected final redd count and support
decision-making.

\textbackslash begin\{table\}

\textbackslash caption\{\label{tab:unnamed-chunk-5}Estimated total
number of Winter-run redds and resulting number of redds that represent
1\% of the population. Estimated total redds are based on current count
and expansion number representing average 2005-2022 expansion.\}
\centering

\begin{tabular}[t]{l|r|r|r}
\hline
Name & Expansion Number & Total Redds & 1\%\\
\hline
Current Count & 1 & 152 & 1.52\\
\hline
Anticipated Expansion & 2 & 304 & 3.04\\
\hline
\end{tabular}

\textbackslash end\{table\}

\hypertarget{chinook-salmon-dewatered-redd-estimates}{%
\subsection{Chinook Salmon Dewatered Redd
Estimates}\label{chinook-salmon-dewatered-redd-estimates}}

As of October 01, 2024, \textbf{6} Winter-run redds have
\textbf{emerged} and \textbf{0} have been \textbf{dewatered}. This
leaves \textbf{11} shallow water redds of concern.

There is no real time data on fall-run redd counts. Estimates are
predicted based on estimated dewatering percentages from USFWS (2006)
and spring-run and fall-run spawn timing based on fresh female carcasses
encountered by week from 2003 through 2023. Emergence timing were
predicted from water temperatures below Keswick in 2018 which most
closely aligns with 2024 operations targeting 53.5 F at Clear Creek.
Fall-run dewatered redd estimates range from \textbf{9.7} to
\textbf{13.3\%}. Note that fall-run dewatering estimates are likely
overestimated using the dewatering percentages from USFWS (2006), and
likely do not reflect actual dewatering percentages and should only be
used for comparative purposes between scenarios. A comparative analysis
between field and modeled dewatering percentages by Gosselin and Beer
(2024) can be found here:
\url{https://www.cbr.washington.edu/sacramento/fishmodel/Note_on_Redd_Dewatering_Observed_v_Predicted.pdf}.

\hypertarget{carryover-effects-to-next-year-winter-run-brood}{%
\subsection{Carryover Effects to Next Year Winter-run
Brood}\label{carryover-effects-to-next-year-winter-run-brood}}

An analysis on the relationship between winter-run chinook salmon
temperature dependent mortality relationship and Shasta Reservoir
end-of-year storage suggests a threshold of 2,200 TAF end of September
Shasta Storage to assess the impacts of TDM impacts on next year's
cohort. Next year's cohort is expected to experience minimal TDM impacts
when end of September Shasta Storage is greater than this threshold,
while values lower than 2,200 TAF are correlated with potentially more
negative TDM impacts. As of August, End of September Shasta Storage is
expected to be \textbf{2717} TAF.All proposed scenarios are anticipated
to have EOS storage greater than the 2200 TAF threshold and therefore
would not be expected to contribute to TDM impacts to winter-run chinook
salmon in the subsequent year (see Table 2).

\hypertarget{preliminary-predicted-results}{%
\subsection{Preliminary Predicted
Results}\label{preliminary-predicted-results}}

\begin{table}

\caption{\label{tab:unnamed-chunk-8}Summary of water volume and winter-run and fall-run dewatering estimates related to flow scenarios. Each scenario uses actual flow-to-date as of most current report and proposed flows for the remainder of the incubation period. Percentage of the population lost is based on the September 10, 2024 count of 152 Winter-run redds. See Scenario Descriptions file for additional information on each scenario.}
\centering
\fontsize{8}{10}\selectfont
\begin{tabular}[t]{>{\raggedright\arraybackslash}p{4cm}|l|l|l|l|l|l}
\hline
Metric & aug90wradjdec & aug90wrfr6000 & aug90wrfr6500 & aug90wrshape2dec & aug90wrshapedec & spgoct3\\
\hline
Avg Sept Flow (cfs) & 8358 & 8358 & 8358 & 8358 & 8358 & 8358\\
\hline
Avg Oct Flow (cfs) & 6983 & 6273 & 6628 & 6999 & 6999 & 6870\\
\hline
Sept-Feb Total Volume (TAF) & 1980 & 1964 & 1975 & 1940 & 1940 & 1932\\
\hline
Aug-Sept Total Volume (TAF) & 1213 & 1213 & 1213 & 1213 & 1213 & 1213\\
\hline
Anticipated EOS Storage (TAF) & 2717 & 2717 & 2717 & 2717 & 2717 & 2717\\
\hline
Winter-run Redds Dewatered & 2 & 2 & 2 & 1 & 1 & 1\\
\hline
Winter-run Percent Lost (current count) & 1.32 & 1.32 & 1.32 & 0.66 & 0.66 & 0.66\\
\hline
Winter-run Percent Lost (mean expansion of 2) & 0.66 & 0.66 & 0.66 & 0.33 & 0.33 & 0.33\\
\hline
Winter-run Redds Dewatered (250 cfs buffer) & 2 & 2 & 2 & 1 & 1 & 1\\
\hline
Winter-run Percent Lost (250 cfs buffer) & 1.32 & 1.32 & 1.32 & 0.66 & 0.66 & 0.66\\
\hline
Fall-run dewatered (\%) & 13.3 & 9.7 & 12.2 & 10.6 & 10.6 & 10.6\\
\hline
\end{tabular}
\end{table}

\begin{figure}
\centering
\includegraphics{Real-time-Estimates_Prelim_automated_v5_files/figure-latex/unnamed-chunk-9-1.pdf}
\caption{Actual or estimated emergence dates of SRWC redds and actual or
estimated dewatering flow for the September-October estimated redd
emergence dates as compared to Keswick flow (in cfs) of proposed
management alternatives. Points represent dewatered (De), emerged (Em),
or remaining (Re) redds. Numbers inside of points indicate how many
redds share that estimated emergence date and actual/estimated
dewatering flow. Points that fall above/to the right of a flow
alternative line are expected to be dewatered given that management
alternative is followed. Points that fall below/to the left of/on a flow
alternative line are not expected to be dewatered, given that management
alternative is followed. Shaded gray box shows period of real-time flow
data; dashed black line equals KWK gauge flow and solid black line
equals KES flow (from
\href{https://www.cbr.washington.edu/sacramento/data/query_river_table.html}{SacPas}).}
\end{figure}

\begin{table}

\caption{\label{tab:unnamed-chunk-10}Description of scenarios being considered and compared by the Upper Sacramento Scheduling Team. Scenario name includes the shorthand notion in parentheses for cross-referencing with graph and tables.}
\centering
\begin{tabular}[t]{>{\raggedright\arraybackslash}p{3cm}|>{\raggedright\arraybackslash}p{13cm}}
\hline
Scenario & Description\\
\hline
Aug 90\% WR shape dec (aug90wrshapedec) & Developed on 9/6/2024. Based on the 90\% forecast exceedance. Follows ramping rates.  Shifts 500cfs diversion from late Oct to early Oct.\\
\hline
Aug 90\% WR shape2 dec (aug90wrshape2dec) & Developed on 9/11/2024. Based on the 90\% forecast exceedance. Follows ramping rates.  Shifts 500cfs diversion from late Oct to early Oct.  Removes end of Sept 4-day flow reduction at Keswick\\
\hline
SPG Oct 3 (spgoct3) & Scenario developed on 10/7/2024 based on Shasta Planning Group advice transmitted to USST on October 3. Reduce flows to 6,750 cfs (rounded to 6,800 cfs) after heat wave (Oct 12), then try to reduce releases after  winter-run redd (ID 4152-24-W) emerges Oct 24. This scenario is similar to Aug 90\% WR shape2 dec.  Assumes unable to reduce end of Oct but begins rampdown Nov 1.\\
\hline
\end{tabular}
\end{table}

\hypertarget{references}{%
\section{References}\label{references}}

Gard, Mark. 2006. Relationships between flow fluctuations and redd
dewatering and juvenile stranding for Chinook Salmon and Steelhead in
the Sacramento River between Keswick Dam and Battle Creek. 94 pages.

Gosselin, J.L. and W.N. Beer. 2024. Sacramento River Winter-run Chinook
Salmon Redd Dewatering: a Note on Comparing Observed and Predicted.
Central Valley Prediction and Assessment of Salmon (SacPas;
\url{https://www.cbr.washington.edu/sacramento/}). Columbia Basin
Research, School of Aquatic and Fishery Sciences, University of
Washington.

\end{document}
